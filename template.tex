%! TeX root = template.tex

% !TEX TS-program = xelatex
% !TEX encoding   = UTF-8 Unicode

\documentclass[12pt, a4paper]{article}

\usepackage{amsxtra, amsfonts, amsthm, amssymb, mathtools}
\usepackage{enumitem}
\usepackage{xcolor}
\usepackage{tcolorbox}
\usepackage{tikz}
\usepackage{comment}
\usepackage{pgfplots}
\pgfplotsset{compat = 1.18}
% Theming

% Polar Night

\definecolor{nord0}{HTML}{2E3440}
\definecolor{nord1}{HTML}{3B4252}
\definecolor{nord2}{HTML}{434C5E}
\definecolor{nord3}{HTML}{4C566A}

% Snow Storm

\definecolor{nord4}{HTML}{D8DEE9}
\definecolor{nord5}{HTML}{E5E9F0}
\definecolor{nord6}{HTML}{ECEFF4}

% Frost

\definecolor{nord7}{HTML}{8FBCBB}
\definecolor{nord8}{HTML}{88C0D0}
\definecolor{nord9}{HTML}{81A1C1}
\definecolor{nord10}{HTML}{5E81AC}

% Aurora

\definecolor{nord11}{HTML}{BF616A}
\definecolor{nord12}{HTML}{D08770}
\definecolor{nord13}{HTML}{EBCB8B}
\definecolor{nord14}{HTML}{A3BE8C}
\definecolor{nord15}{HTML}{B48EAD}





\title{Template}
\author{Name}
\date{}

\begin{document}

\maketitle
\newpage

\section{Definition of Limit}

\begin{tcolorbox}[colframe=nord11!, title=$\varepsilon - \delta$ definition of limit]
Let $f$ be a function defined on an open interval containing $c$ (except possibly at $c$), and let $L$ be a real number. The statement
$$
\lim_{x\to\infty} f(x) = L
$$
means that for each $\varepsilon > 0$ there exists a $\delta > 0$ such that if
$$
0 < \left| x-c \right| < \delta
$$
then
$$
\left| f(x) - L \right| < \varepsilon
$$
\end{tcolorbox}

\bigskip

\begin{tcolorbox}[colframe=nord9, title=\textbf{THEOREM 2.4} The Constant Multiple Rule]
If $f$ is a differentiable function and $c$ is a real number, then $cf$ is also differentiable and
$$
\frac{d}{dx}\left[ cf(x) \right] = cf'(x)
$$


\end{tcolorbox}

\end{document}
