%! TeX root = template.tex

% !TEX TS-program = xelatex
% !TEX encoding   = UTF-8 Unicode

\documentclass[12pt, a4paper]{article}

\usepackage{amsxtra, amsfonts, amsthm, amssymb, mathtools}
\usepackage{enumitem}
\usepackage{xcolor}
\usepackage{pagecolor}
\usepackage{bookmark}
\usepackage[margin = 1.5cm]{geometry}

\usepackage[most]{tcolorbox}

\usepackage{tikz}
\usepackage{comment}

\usepackage{pgfplots}
\pgfplotsset{compat = 1.18}

\usepackage{hyperref}

% Theming

% Polar Night

\definecolor{nord0}{RGB}{46, 52, 64}
\definecolor{nord1}{RGB}{59, 66, 82}
\definecolor{nord2}{RGB}{67, 76, 94}
\definecolor{nord3}{RGB}{76, 86, 106}

% Snow Storm

\definecolor{nord4}{RGB}{216, 222, 233}
\definecolor{nord5}{RGB}{229, 233, 240}
\definecolor{nord6}{RGB}{236, 239, 244}

% Frost

\definecolor{nord7}{RGB}{143, 188, 187}
\definecolor{nord8}{RGB}{136, 192, 208}
\definecolor{nord9}{RGB}{129, 161, 193}
\definecolor{nord10}{RGB}{94, 129, 172}

% Aurora

\definecolor{nord11}{RGB}{191, 97, 106}
\definecolor{nord12}{RGB}{208, 135, 112}
\definecolor{nord13}{RGB}{235, 203, 139}
\definecolor{nord14}{RGB}{163, 190, 140}
\definecolor{nord15}{RGB}{180, 142, 173}

\pagecolor{nord6!}
\color{nord0!}

\hypersetup{
  colorlinks = true,
  linkcolor = nord0
}


\newtcolorbox{theorembox}[1][]{colback = nord6!, colframe = nord11!, title={\textbf{#1}}}
\newtcolorbox{definitionbox}[1][]{colback = nord6!, colframe = nord12!, title={\textbf{#1}}}
\newtcolorbox{examplebox}[1][]{colback = nord6!, colframe = nord10!, title={\textbf{#1}}}


\begin{document}

\begin{theorembox}[Theorem 2.4]
If $f$ is a differentiable function and $c$ is a real number, then $cf$ is also differentiable and
$$
\frac{d}{dx}\left[ cf(x) \right] = cf'(x)
$$
\end{theorembox}

\bigskip

\begin{definitionbox}[$\varepsilon - \delta$ definition of limit]
Let $f$ be a function defined on an open interval containing $c$ (except possibly at $c$), and let $L$ be a real number. The statement
$$
\lim_{x\to\infty} f(x) = L
$$
means that for each $\varepsilon > 0$ there exists a $\delta > 0$ such that if
$$
0 < \left| x-c \right| < \delta
$$
then
$$
\left| f(x) - L \right| < \varepsilon
$$
\end{definitionbox}

\bigskip


\begin{examplebox}[Example 1. Evaulating Basic Limits]
$$
\lim_{x\to\infty} 3 = 3
$$
\end{examplebox}

\end{document}
